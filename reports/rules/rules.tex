\documentclass[10pt]{article}
\usepackage{hyperref}
\usepackage{mathpartir}

\newcommand{\GammaH}{\widehat{\Gamma}}

% houtje touwtje manier om typwerk te besparen
\newcommand{\trule}[3]{
	\begin{mathpar}
		\inferrule
			{#1}
			{#2}
			\hspace{1cm}
			{\textbf{\textrm{[#3]}}}
	\end{mathpar}
}

% provides Gamma |- UL
\newcommand{\GUL}{\GammaH \vdash_{UL}}

\newcommand{\letin}[2]{\textbf{\textrm{~let~}} #1 \textbf{\textrm{~in~}} #2}
\newcommand{\lam}[2]{\textbf{\textrm{~\textbackslash}}#1 \rightarrow #2} % Is dit handig met die rightarrow als scheiding voor een lambda?

\begin{document}

% TODO: checken van alles, want volgens mij klopt het nog niet allemaal
\trule
	{}
	{\GUL c : \tau}
	{t-const}

\trule
	{\GammaH (x) = \tau}
	{\GUL x : \tau}
	{t-var}

\trule
	{\GUL e_1 : \tau_1 \\ \GammaH[x \mapsto \tau_1] \vdash e_2 : \tau}
	{\GUL \letin{x = e_1}{e_2} : \tau}
	{t-let}

\trule
	{\GUL x : \tau_1 \\ \GammaH[e \mapsto \tau_1] \vdash e : \tau}
	{\GUL \lam{x}{t} : \tau}
	{t-lam}

\end{document}
