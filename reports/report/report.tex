\documentclass[10pt]{article}
\usepackage{hyperref}

\title{Documentation for Automatic Program Analysis project 2}
\author{Frank Dedden (3705269) \and Wilco Kusee (3800296)}


\begin{document}
\maketitle

\section{Architecture}
In the \texttt{analysis/src} directory we can find five files:
\begin{itemize}
	\item \texttt{Expr.hs}
	\item \texttt{Lexer.hs}
	\item \texttt{Main.hs}
	\item \texttt{Parser.hs}
	\item \texttt{Typechecker.hs}
\end{itemize}
The files \texttt{Lexer.hs} and \texttt{Parser.hs} implement a lexer/parser combination using the Parsec library. This parses a limited subset of the haskell syntax into a list of statements. These statements are defined in \texttt{Expr.hs}.

\texttt{Typechecker.hs} provides a type checker, that creates a list of constraints. These constraints can then be solved using the solver that can be found in \texttt{Main.hs}. \texttt{Main.hs} ties everything together and passes the contents of a file (given as a commandline argument) to the correct functions.

\section{Type and constraint systems}
We do typechecking using Algorithm W, and use a two-stage approach where we first gather constraints and later on solve and apply them. This solving is done by our constraint solver. The rules the complete system can be found in \texttt{analysis/reports/rules.pdf}

\section{Example}
We provided 3 simple examples. These can be found in the \texttt{analysis/examples} directory.

Example number 1 shows a data constructor, a few function declarations and usage of the case statement. Example 2 shows a data constructor and a if-then-else conditional. Example 3 provides a simple showcase of the let binding and lambda expressions.

\section{Features and limitations}
Our lexer/parser combination supports:
\begin{itemize}
	\item A lambda-calculus based language
	\item Let-bindings
	\item Condional expression (if-then-else and case)
	\item Function definition and application
	\item Binary operators
	\item Data constructors
\end{itemize}
The type checker supports all these language constructions, except for the data constructors and binary operators.

Unfortunately however, the constraint solver does not work. This is due to an unfinished implementation of the unification algorithm, as we had troubles implementing this. The sourecode of ths function can be found in \texttt{Main.hs}, but is not used.

\end{document}
